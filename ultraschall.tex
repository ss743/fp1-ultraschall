\documentclass[12pt]{article}

\usepackage{fancyhdr}
\usepackage{geometry}
\usepackage{ucs}
\usepackage[utf8x]{inputenc}
\usepackage[T1]{fontenc}
\usepackage[ngerman]{babel}
\usepackage{amsmath,amssymb,amstext}
\usepackage{hyperref}
\usepackage{cancel}
\usepackage{dsfont}
\usepackage{physics}
\usepackage{lmodern}
\usepackage{enumerate}
\usepackage{enumitem}
\usepackage{graphicx}
\usepackage{listings, color}
\usepackage[labelfont=bf]{caption}
\usepackage{titling}

\lstset{basicstyle=\scriptsize} %Quellcode mit Umlauten und ganz klein
\lstset{literate=
  {Ö}{{\"O}}1
  {Ä}{{\"A}}1
  {Ü}{{\"U}}1
  {ß}{{\ss}}2
  {ü}{{\"u}}1
  {ä}{{\"a}}1
  {ö}{{\"o}}1
}


%Geometrie----------------------------------------------------------------------------------------------------------

\geometry{a4paper, top=25mm, left=15mm, right=15mm, bottom=25mm,headsep=10mm, footskip=10mm}
\pagestyle{fancy}
\setlength{\parindent}{0pt} %Zeileneinrückung

\fancyhf{} %Setzt voreingestellte Kopf-und Fußzeilen-Eigenschaften zurück

\lhead{\nouppercase{\leftmark}}
\chead{}
\rhead{\thepage}

\lfoot{}
\cfoot{}
\rfoot{}

\title{\vspace{0cm}{\Huge Fortgeschrittenen-Praktikum I:\\ \vspace{1cm} Ultraschall}}
\author{Saskia Bondza\\Simon Stephan}
\date{durchgeführt am 20./21.09.2016}

\pretitle{%
  \begin{center}
  \LARGE
  \includegraphics[width=6cm,]{figures/siegel}\\[\bigskipamount]
}
\posttitle{\end{center}}

%neue Commands----------------------------------------------------------------------------------------------------------
\newcommand{\nab}{\vec{\nabla}} %direkter Befehl mit Vektorpfeil
\newcommand{\gra}[3][0.7]{
	\begin{minipage}[h!]{\textwidth}
		\centering
		\includegraphics[width=#1\textwidth]{figures/#2.png}
		\captionof{figure}{#3}
	\end{minipage}
	\vskip 30 pt
}
\newcommand{\graTwo}[4][0.5]{
	\begin{minipage}[h!]{\textwidth}
		\centering
		\includegraphics[width=#1\textwidth]{figures/#2.png}
		\includegraphics[width=#1\textwidth]{figures/#3.png}
		\captionof{figure}{#4}
	\end{minipage}
	\vskip 30 pt
}
\newcommand{\code}[1]{\texttt{#1}}


%Titel,Inhalt----------------------------------------------------------------------------------------------------------

\begin{document}
\pagenumbering{gobble} %verstecke Seitenzahl
\maketitle
\newpage

\thispagestyle{empty}
\section*{Abstract}


Im Rahmen dieses Versuches werden Grundlagen der Beugung an Amplituden- und Phasengittern und der Fourieroptik vermittelt. 
Hierzu werden die Gitterkonstanten eines Sinusgitters sowie verschiedener Amplitudengitter an Hand der Beugungsmaxima bestimmt. Zudem werden für die Amplitudengitter das Auflösungsvermögen und für eines dieser Gitter die Aperaturfunktion und Verhältnis der Spaltbreite zur Gitterkonstante bestimmt. 
Die wesentlichen Teile des Versuchsaufbaus, der verwendet wird, sind dabei eine monochromatischen, kohärenten Lichtquelle (ein Helium-Neon-Laser), die nach Durchgang durch eine Blende auf das Gitter (bzw. Ultraschallzelle) trifft, ein rotierender Motorspiegel der das zeitliche Signal in ein räunliches umwandelt, sowie an ein Oszilloskop angeschlossene Photodioden. Außerdem werden verschiedene Linsen und Spiegel verwendet um den Strahlengang geeignet zu lenken.
Unsere Ergebnisse...
\newpage
\tableofcontents
\newpage

%Schreiben----------------------------------------------------------------------------------------------------------
\pagenumbering{arabic} %verstecke Seitenzahl
\section{Einleitung}




\newpage
\section{Theoretische Grundlagen}
\newpage
\section{Sinusgitter}

\newpage
\subsection{Versuchsaufbau und -Durchführung}





\newpage
\subsection{Auswertung}


\newpage
\subsection{Diskussion}

\newpage
\section{Amplitudengitter}

\newpage
\subsection{Versuchsaufbau und -Durchführung}





\newpage
\subsection{Auswertung}


\newpage
\subsection{Diskussion}

\section{Ultraschall-Phasengitter}

\newpage
\subsection{Versuchsaufbau und -Durchführung}





\newpage
\subsection{Auswertung}


\newpage
\subsection{Diskussion}

\newpage
\section{Zusammenfassung und Diskussion}

\subsection{Zusammenfassung der Ergebnisse}



\subsection{Diskussion}


\newpage
\section{Anhang} 


\subsection{Laborheft}
%\begin{minipage}{\textwidth}
%\centering
%\includegraphics[width=0.9\textwidth]{figures/IMG_20151002_141014.jpg}
%\end{minipage}

\newpage
\listoffigures

%Literatur----------------------------------------------------------------------------------------------------------

%\cite{les}
\newpage
\thispagestyle{empty}
\begin{thebibliography}{9}

%\bibitem{staat}
%  Tobijas Kotyk,
%  \emph{Versuche zur Radioaktivität im Physikalischen Fortgeschrittenen Praktikum an der Albert-Ludwigs-Universität Freiburg},
%  Albert-Ludwigs-Universität, Freiburg,
%  2005
  

  
%\bibitem{molmasse}
%  \emph{http://www.convertunits.com/molarmass/<ELEMENTNAME AUF ENGLISCH>}, Stand 28.09.2015
  
\bibitem{bibhall}
\emph{http://www.schule-bw.de/unterricht/faecher/physik/online\_material/e\_lehre\_2/teilchenfeld/\\halleffekt.htm}, Stand 15.09.2016

\bibitem{anleitung}
\emph{http://hacol13.physik.uni-freiburg.de/fp/Versuche/FP1/FP1-8-Kernspinresonanz/Anleitung.pdf}, Stand 15.09.2016

\bibitem{Babykatze}
\emph{http://hacol13.physik.uni-freiburg.de/fp/Versuche/FP1/FP1-8-Kernspinresonanz/Anhang/A.Klett.pdf}, Stand 19.09.2016
\end{thebibliography}

\end{document}